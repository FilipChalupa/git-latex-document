\documentclass{article}
\usepackage[utf8]{inputenc}

\title{Kolaborativní tvorba LaTeX dokumentů s podporou Gitu, vedoucí Ing. Tomáš Kalvoda, Ph.D. (kalvotom@fit.cvut.cz)}
\author{Filip Chalupa (chalufil@fit.cvut.cz)}
\date{February 2017}

\usepackage{natbib}
\usepackage{graphicx}

\begin{document}

\maketitle

\section{Klíčová slova}
LaTeX 

"LaTeX je komplexní systém pro sazbu dokumentů. Stará se tedy o zalamování textu do odstavců, vkládání obrázků, seznamů, nadpisů, matematických výrazů, rozdělování dokumentu na jednotlivé stránky, ligatury, "vdovy" a "sirotky", odsazení..." \citep{cit1}

Git

Verzovací systém s podporou více uživatelů a ukládáním na vzdálený server například v internetu.

Verzování

Sledování projektu s průběžným ukládání změn pro zpětné dohledání přidaných či odebraných řádků v textovém souboru, případně přidání či odebrání binárního souboru.

Kolaborace

Spolupráce více osob na jednom projektu.

Přístupnost

Umožnění práce se složitějšími nástroji pro osoby s nedostatečnou znalostí ovládání.

\section{Úvod}

Častým scénářem při tvorbě různých skript, informačních podkladů, prezentací je spolupráce několika osob. V takových týmech je důležitá komunikace mezi jednotlivými členy a sdílení aktuálního rozpracovaného díla. K dispozici je spousta nástrojů, jichž smyslem je takovou tvorbu usnadnit; velké množství, různorodost, a v některých případech větší komplexita však značně komplikuje jejich nasazení. Důsledkem je uchýlení se k více známým a z podstaty primitivnějším řešením, například přeposílání rozdělané práce přes e-mail nebo předávání podkladů přes externí uložiště a nedělání záloh předešlých verzí, tj. neverzování. Tým tím ztrácí na efektivitě, což se může podepsat na kvalitě díla, jeho zdlouhavém vyhotovení a v extrémních případech hrozí i ztráta celé práce.

\section{Cíle}
Tristní situaci přednesenou v úvodu je tedy žádoucí vyřešit buď vyškolením účastníků týmu, nebo představením nového nástroje, který disponuje funkcemi komplexních nástrojů, ale zároveň je jednoduchý na použití i pro začátečníka, což je i cílem této práce, zhotovení nové aplikace určené pro operační systémy primárně Windows a Linux. K jeho dosažení je potřeba provést podrobnější analýzu současného stavu. Je tedy nutné proniknout do pracovních postupů autorů, jimž práce v týmu psaním v LaTeXu způsobuje potíže. Na základě například dotazníků či rozhovorů. Tímto se dostaneme do druhé fáze, kdy bude potřeba navrhnout principy aplikace, její nároky na uživatelské rozhraní a podporované funkce. Dále dojde k implementaci vybraného návrhu a na závěr k otestovaní dotazovanými z analytické části.

\section{Současný stav}
Jako ověření, že vytváření nové aplikace má význam, byly prozkoumány současné možnosti, které si kladou za cíl řešení stejného problému. K dispozici jsou například online nástroje ShareLaTeX a Overleaf. První zmíněný, "Online LaTeX editor snadný k použití s možností spolupráce více autorů" \citep{cit2}, se sice hned na úvodní stránce chlubí překladem do češtiny, který je ale nekompletní a práce s ním je tak často matoucí, z textů není jasné, co se zrovna děje, co se od uživatele očekává a jakým způsobem má pracovat. K tomu vyžaduje vysoký měsíční poplatek \citep{cit3} za zpřístupnění většiny funkcí. Druhá služba, Overleaf, překlad do češtiny nenabízí vůbec, což není nutné považovat za nedostatek, protože požadavkem není řešení konkrétně pro české prostředí. Chybí však funkce pro procházení změn v historii projektu.

Obě služby trpí omezením, že se nelze dostat k úpravám projektu mimo ně. Uživatel nemůže tvořit ve svém oblíbeném editoru, musí si vystačit s online editorem, který navíc nenabízí žádnou offline variantu.

Alternativním přístupem je třeba verzovací systém Git, jež je zejména pro začátečníky s tímto systémem příliš komplikovaný.

\section{Řešení}
Řešením je nová aplikace, která je schopná běžet na různých cílových zařízení, se systémem Windows nebo Linux, tak, aby vystihla různorodost zařízení členů týmu (systém, editor LaTeXu), ale zároveň nenutila všechny členy k jejímu aktivnímu využívání. Čtyřčlenný tým může vypadat třeba tak, že jeden autor pracuje na svém počítači se systémem Windows s pomocí nové aplikace, podobně další dva členové se systémem Linux a čtvrtý člen s pokročilou znalostí verzovacího systému Git nemusí novou aplikaci používat vůbec a stejně může bez omezení spolupracovat se zbytkem týmu. Aplikace tedy musí být schopná komunikovat se systémem Git, který splňuje požadavek na zálohování a verzování.

\section{Závěr}
Bakalářská práce je momentálně již rozpracovaná. Byl proveden průzkum mezi šesti učiteli, kteří v LaTeXu běžně tvoří obsah v týmu. V rozhovorech se všichni shodli, že aktuální stav není ideální a až na jednoho člověka by všichni ocenili lepší, přehlednější řešení. Mimo to byly také prozkoumány alternativní nástroje, které jsou ve spojení LaTeXu a spolupráce známé, ale nikdo z tázaných s nimi neměl zkušenost. Konkrétně se jednalo o webové služby ShareLaTeX [2] a Overleaf \citep{cit4}. Závěrem je potvrzení, že má smysl nové řešení vyvíjet a zároveň došlo k jeho detailnější specifikaci. 

\bibliographystyle{plain}
\bibliography{references}
\end{document}
