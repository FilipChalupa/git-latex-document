\chapter{Závěr}

Na základě analýzy problému kolaborace vznikla aplikace pro různé operační systémy, jež si klade za cíl usnadnění práce v týmu zejména na projektech psaných v \LaTeX{u}. Aplikace je spustitelná a otestovaná na cílených platformách. Splňuje stanovené požadavky až na několik vyjímek, které mohou způsobit nepoužitelnost pro některé uživatele. Pro tyto problémy na základě zkoumání byly navrženy způsoby jejich řešení, které je možné využít jako osnovu pro další rozvoj.

Pro demostraci aspoň částečného fungování současného stavu aplikace byla tato bakalářská práce psána s jeho pomocí. Výsledek je verzovaný ve vlastním repozitáři\footnote{\url{https://github.com/Onset/git-latex-document}}. Hlavní nedostatky se týkají autentizace uživatelů na vzdálených serverech a chybějící průvodce rozhraním aplikace, který by vysvětlil, jak ji používat.

Kromě implementace samotné a návrhů na rozšíření byl popsán i proces, jenž ke vzniku aplikace vedl, počínaje formulací problému, rozhovory s cílovou skupinou uživatelů, rozborem existujících řešení, volbou nástrojů pro alternativní řešení, návrhem uživatelského rozhraní, přípravou distribuce, testováním a konče výčtem poznatků při práci s knihovnout NodeGit.

Zejména velký přínost měla pro jejího autora. Díky ní měl možnost například prohloubit své znalosti JavaScriptu a Gitu, vyzkoušel si vytvořit první desktopovou aplikaci a naučil se pracovat se službami pro průběžnou integraci.