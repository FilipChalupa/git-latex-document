\section{Uživatelé}

Formou osobních rozhovorů proběhly diskuze s~pěti učiteli z~ČVUT zaměřujících se převážně na tvorbu materiálů s~tématikou matematiky. Každá diskuze byla zahájena stručným představením a dále otázkami, jaké materiály učitel tvoří a za pomoci jakých nástrojů. Zbytek se odvíjel podle odpovědí na tyto první dva dotazy. Volba respondentů vycházela z~doporučení vedoucího této bakalářské práce a z~rohodnutí autora nevyhledávala další, protože všechny odpovědi hypotézu problému podporovaly a zároveň byly dostatečně různorodé. Následuje výčet poznatků od všech pěti respondentů z~prosince roku 2015.

\subsection{Respondent A}

\begin{itemize}
	\item Ing. Daniel Dombek, Ph.D.
	\item Odborný asistent na katedře aplikované matematiky \cite{kam},
	\item píše zejména handouty pro předmět BI-ZDM\footnote{\url{https://edux.fit.cvut.cz/courses/BI-ZDM/start}},
	\item práci se spoluautory sdílí přes fakultní GitLab\footnote{\url{https://gitlab.fit.cvut.cz/}},
	\item poznámky a další komunikaci posílá přes e-mail,
	\item běžně pracuje sám či až se čtyřmi lidmi,
	\item zálohuje, verzuje i práci, na které dělá jediný,
	\item se spolupracujími je domluvený, že nemění v~jednu chvíli stejné soubory kvůli vyvarování se možných kolizí,
	\item píše v~TeXstudio\footnote{\url{http://www.texstudio.org/}},
	\item sdílí pomocí nástroje TortoiseGit\footnote{\url{https://tortoisegit.org/}} a pluginu\footnote{\url{https://github.com/Darthholi/WDX\_GitCommander}} pro Git do Total Commanderu,
	\item setkává se s~problémem hromadění tabů, znaků pro odsazení, který je pravděpodobně způsoben různou konfigurací editorů ostatních autorů, kdy možná některý převádí mezery na začátku řádků pro přehledné odsazení zdrojového textu na neúměrný počet tabů,
	\item nevyhovují mu přebytečné neviditelné znaky na koncích řádků,
	\item v~Gitu vytváří jeden commit za svou celodenní práci,
	\item má zájem sledovat historii o~jeden commit zpět.
\end{itemize}


\subsection{Respondent B}

\begin{itemize}
	\item Ing. Štěpán Starosta, Ph.D.
	\item Odborný asistent na katedře aplikované matematiky \cite{kam},
	\item sdílí kancelář s~prvním respondentem, takže je přítomný u~předchozího rozhovoru a ve většině se na odpovědích shodne,
	\item zmiňuje konkurenci Overleaf\footnote{\url{https://www.overleaf.com/}}, ale není jejím aktivním uživatelem,
	\item občas řeší konflikty, když méně zkušený spoluautor v~Gitu zapomene před zahájením své práce stáhnout aktuální úpravy ostatních,
	\item také mimo Git komunikuje přes e-maily,
	\item píše v~Sublime Text s~pluginem GitGutter\footnote{\url{https://github.com/jisaacks/GitGutter}},
	\item má spoustu pomocného kódu, který sdílí napříč různými projekty, ale neví, jak ho efektivně vždy přenášet,
	\item rád používá pro zvýraznění změn Latexdiff\footnote{\url{https://www.ctan.org/pkg/latexdiff}},
	\item chtěl by nějak do \LaTeX{u} dostat poznámky z~Gitu například o~aktuální revizi, aby získal lepší přehled o~již vytištěných verzích a mohl je snadno zpětně spárovat s~historií commitů.
\end{itemize}


\subsection{Respondent C}

\begin{itemize}
	\item Mgr. Michal Kupsa, Ph.D.
	\item Odborný asistent na katedře aplikované matematiky \cite{kam},
	\item píše články o~deseti až dvaceti stranách,
	\item zkušenosti má s~verzovacím nástrojem Git a SVN,
	\item pracuje v~Emacs\footnote{\url{https://www.gnu.org/software/emacs/}},
	\item s~týmem také komunikuje přes e-mail,
	\item ve správci úkolů (issue tracker) na GitLabu mu chybí zobrazování vzorců, protože GitLab je neumí příliš dobře,
	\item možnost přehledu historie vnímá jako pozitivum.
\end{itemize}


\subsection{Respondent D}

\begin{itemize}
	\item Ing. Daniel Vašata, Ph.D.
	\item Odborný asistent na katedře aplikované matematiky \cite{kam},
	\item píše v~Sublime Text s~pluginem LaTeXTools\footnote{\url{https://github.com/SublimeText/LaTeXTools}},
	\item má zkušenost s~Gitem a SVN,
	\item pro pohodlnější práci s~Gitem používá SmartGit\footnote{\url{http://www.syntevo.com/smartgit/}},
	\item commity vytváří na konci dne, což mu vychází na větší logickou část,
	\item zajímá ho historie na úrovni jednotlivých souborů,
	\item je zaujatý nástrojem Latexdiff.
\end{itemize}


\subsection{Respondent E}

\begin{itemize}
	\item RNDr. Ondřej Suchý, Ph.D.
	\item Odborný asistent na katedře teoretické informatiky \cite{kti},
	\item tvoří prezentace, píše články, úlohy pro středoškoláky,
	\item účastní se v~týmech do sedmi lidí,
	\item výjimkou je tým na tvorbu úloh pro středoškoláky, jenž se skládá z~dvanácti lidí,
	\item práci sdílí výhradně přes SVN,
	\item s~\uv{významnějšími} autory přes Dropbox\footnote{\url{https://www.dropbox.com}}, avšak zde mu nevyhovuje správa verzí,
	\item nasazení Gitu vidí jako zbytečnou komplikaci pro již zaběhnuté prostředí,
	\item navíc každý úkon je pro něj v~Gitu dvakrát složitější než v~SVN i kvůli náročnějšímu řešení merge,
	\item komunikuje přes IRC či e-mail, který občas obohatí o~diff export z~SVN,
	\item v~rozdílech mezi jednotlivými commity vyžaduje zvýraznění na úrovni změněných slov, standardní rozlišování pouze řádků je nedostatečné,
	\item píše v~Kile\footnote{\url{http://kile.sourceforge.net/}}, na kterém si chválí automatické doplňování a náhledy.
\end{itemize}

Kromě respondentů byl zdrojem podnětů i samotný vedoucí práce, Ing.~Tomáš Kalvoda,~Ph.D. Odborný asistent na katedře aplikované matematiky, který průběžně usměrňoval vývoj řešení. Přišel s~původní hypotézou.