\section{Existující řešení}

% @TODO: https://cs.sharelatex.com/ a https://www.overleaf.com/
Mezi nástroje specializující se na tvorbu \LaTeX{ových} dokumentů v týmu se specializují zejména dva, ShareLaTeX a Overleaf. Zde je přehled o jejich stavu v prosinci roku 2015.

\subsection{ShareLaTeX}

\begin{itemize}
	\item Úpravy dokumentu provádí uživatel ve webovém prohlížeči,
	\item není možné použít vlastní editor,
	\item je vyžadováno neustále připojení k internetu, bez něho není možné práci ani prohlížet,
	\item částečné přeložení do češtiny, některé texty UI jsou v angličtině, font na úvodní stránce nepodporuje háčky a čárky,
	\item pro spolupráci více lidí je nutnost přejít na placený tarif (\uv{€16.94 every měsíc} \cite{sharelatex-pricing})
	\item při souběžné účasti jsou vidět ve zdrojových textech kurzory všech,
	\item změny se synchronizují téměř okamžitě,
	\item kontrola pravopisu, automatické dokončování, počítání slov,
	\item vše se ukládá přímo na ShareLaTeX, případně lze ručně stáhnout aktuální verzi,
	\item lze procházet verze po jednotlivých změnách, upravené části jsou zvýrazněny barvou,
	\item člověk bez účtu si může zobrazit pouze projekt, který je autory označený jako veřejný,
	\item nové projekty je možné zakládat z předpřipravených šablon.
\end{itemize}


\subsection{Overleaf}

\begin{itemize}
	\item Také nutí uživatele používat pouze webový prohlížeč, nefunguje offline,
	\item většina nabízených funkcí je k dispozici zdarma,
	\item chybí překlad do češtiny,
	\item projekty je možné zakládat z šablon,
	\item podobně jako ShareLaTeX nabízí rychlý náhled vedle editoru,
	\item nezobrazuje kurzory ostatních aktivních spolupracujících,
	\item nové změny se projevují s drobným zpožděním,
	\item pro založení projektu je vyžadován účet, ostatní spolupracovníci ho nepotřebují,
	\item je možné si projekt stáhnout v podobě Gitového repozitáře, který ale neobsahuje podrobnou historii, ale je jeden commit,
	\item také umí kontrolu pravopisu, bez češtiny.
\end{itemize}

Alternativou k těmto nástrojům jsou vlastní řešení. Každý autor může používat svůj oblíbený editor, například TeXstudio, Sublime Text, a práci sdílet pomocí Gitu (GitKraken\footnote{\url{https://www.gitkraken.com/}}, SmartGit), SVN, e-mailu či Dropboxu.