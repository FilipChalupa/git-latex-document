\chapter{Návrhy na rozšíření}

Zejména kvůli nedostatkům pro tuto práci klíčové knihovny NodeGit nebylo možné v omezeném čase postihnout veškeré neduhy výsledné aplikace, které byly objeveny během vývoje či při uživatelském testování. Následující přehled by měl navrhnout jejich řešení, případně doplnit Aplikaci o další funkcionalitu, která by aplikaci významně prospěla.


\section{Zlepšení stávajících funcionalit}

\subsection{Nezobrazovat příliš obsáhlé změny}

U ručně psaných textů se jedná sice spíše o okrajový problém, avšak u projektů, kde uživatel přidává do projektů například celé knihovny či generované texty, se může stát, že množství těchto textů bude příliš náročné na vypsání do okna aplikace například v detailu commitu. Taková zátěž může způsobit nežádoucí pomalý běh aplikace či její úplné zamrznutí. Bylo by tedy vhodné takový situací předcházet například u větších změn zobrazovat jen kratší název a zbytek jen na vyžádání uživatele například stiskem tlačítka „zobrazit více“.

\subsection{Zobrazovat celou historii}

Obrazovka s historií projektu nyní zobrazuje jen commity do maximálního počtu 250. Tento limit zaručuje u projektů s bohatou historií plynulý chod za cenu toho, že uživatel starší commity nevidí. Vzhledem k tomu, že plynulost je potřeba zachovat, bylo by vhodné uživateli nabídnout aspoň stránkování, aby se nevypisovaly všechny commity naráz, nebo implementovat chytřejší vykreslování, které budí dojem souvislé historie i přes omezení množství commitů, které mohou být v jeden okamžit načteny do grafického rozhraní, které je tím nejslabším článkem při práci s větším množstvím dat.


Upozorňovat na commit  podle jazyka (php vs LaTeX)
Přidat více vysvětlivek
Lepší/chytřejší průvodce přidáváním projektu
Diff - problémy s kódováním
Při integrování zobrazit commity


\section{Nové funkcionality}
Možnost měnit e-mail/jméno
Syntax highlighting
Přepínání větví
Tagování
Odkazovat na web (hash na github detail)
Možnost rovnou otevřít vlastní editor
Detekce, že uživatel má všechny závislosti (Cred. manager, Pageant)
Hlášení chyb
Zobrazování obrázků v diffu


\section{Ostatní zlepšení}
Build macOS
