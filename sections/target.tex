\chapter{Cílová skupina uživatelů}
Záměrem vzniklé aplikace je vytvoření prostředí pro snadnou správu, tj. sdílení a verzování, textových dokumentů.
% TODO: co je to textový dokument
Důraz je kladen na formát LaTeXu, avšak nejedná se o omezení. Aplikace funguje jako nadstavba nad systémem Git, takže umožňuje spolupráci více autorů, kde každý nemusí aplikaci používat. Vzhledem k tomu, že Git je velmi populární nástroj,
% @TODO: doplnit nějakou statistiku
někteří autoři dokumentů ho na sdílení a verzování použíbají již teď. Pro snadnou práci v libovolném týmu, kde se mohou nacházet uživatelé s i bez aplikace, je tento scénář podporován. Není vytvořeno ani žádné omezení na službu, která sprostředkovává funkci vzdáleného serveru.
% @TODO: doplnit referenci, co je vzdálený server
Je tedy možné využít například veřejný Github, Bitbucket či třeba provozovat GitLab na vlastním serveru. Vzdálený server slouží jako prostředník mezi autory, kdy na něj jsou ukládány průběžně verze všech autorů. Libovolný autor tak může přejímat aktuální sdílenou verzi a dále ji rozšiřovat. Není však po úplně celou dobu závislý na připojení k internetu, tj. na připojení ke vzdálenému serveru. Připojení je pouze vyžadováno v okamžiku, kdy autor chce nahrát své změny, poskytnout ostatním autorům a zároveň zazálohovat, nebo stáhnout nové úpravy ostatních.
% TODO: vysvětlit, co jsou to změny
Každopádně i bez připojení má autor lokálně uloženou poslední známou historii změn, na které může pracovat. S další synchronizací se pak rozhodne, jestli své změny poskytne i ostatním. Může se totiž stát, že v době bez spojení například někdo rozvinul dokument jiným směrem a místní změny tak nemusí být dále užitečné nebo nemusí být kompatibilní s aktualizovaným stavem dokumentu. V takové situaci aplikace vyzve autora k vyřešení takzvaných merge confliktů, kdy je potřeba některé změny doupravit.
