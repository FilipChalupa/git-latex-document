\chapter{Distribuce}

Aplikaci je možné na cílové zařízení nainstalovat dvěma způsoby. Prvním je stažení například pomocí Gitu ve formě zdrojových kódů, které jsou dostupné v~repozitáři Onset/FitGit\footnote{\url{https://github.com/Onset/FitGit}} na GitHubu. Pro kompilaci a spuštění zdrojových kódů je potřeba nainstalovat Node.js\footnote{\url{https://nodejs.org/}} a Git\footnote{\url{https://git-scm.com/}}, případně nástroje pro kompilaci jazyka C. Tento postup je shodný pro platformu Windows, macOS a linuxové distribuce. Kompilace a následné spuštění se poté provádí tako:

\FloatBarrier
\begin{listing}[ht]
	\begin{minted}[]{bash}
# stáhne repozitář na lokální disk
git clone git@github.com:Onset/FitGit.git
cd FitGit

# nainstaluje některé závislosti pro běh Aplikace
npm install

# nainstaluje další závislosti a zkompiluje knihovnu NodeGit
npm run prebuild

# v adresáři dist vytvoří spustitelnou aplikaci
npm run build-local
	\end{minted}
	\caption{Spuštění ze zdrojového kódu}
\end{listing}
\FloatBarrier

Zmíněný nástroj \mintinline{bash}{npm} je součástí Node.js a slouží ke správě JavaScriptových závislostí a balíčků.

Druhou variantou je možnost využít již předkompilovaných balíčků pro Windows a Linux, které pro běh také vyžadují nainstalovaný Git. Balíčky jsou opět k~dispozici v~repozitáři projektu, v~tomto případě pod záložkou releases\footnote{\url{https://github.com/Onset/FitGit/releases}}. Nové jsou automaticky vytvářeny ke každé otagované verzi, o~které se strají nástroje pro průběžnou integraci Appveyor\footnote{\url{https://www.appveyor.com/}} a Travis CI\footnote{\url{https://travis-ci.org/}}. Konfigurace se nachází v~kořenovém adresáři repozitáře aplikace v~souborech \path{appveyor.yml}\footnote{\url{https://github.com/Onset/FitGit/blob/master/appveyor.yml}} a \path{.travis.yml}\footnote{\url{https://github.com/Onset/FitGit/blob/master/.travis.yml}}.

Tyto dva nástroje byly zvoleny kvůli možnosti zdarma automaticky spouštět skládání testovacích sestavení (buildů), případně ostrých sestavení pro uživatele. Appveyor pak kvůli specializaci na systémy s~Windows a Travis na~systémy unixové.

% @TODO: ref na kód
Appveyor i Travis jsou propojeni s~repozitářem FitGit na GitHubu. S~každým novým nahráním změn zdrojových kódů se tyto dva nástroje spustí. Nahrají si k~sobě aktualizovaný repozitář, pro který provedou příkazy uvedené v~prvním způsobu instalace. Pokud jsou změny označeny tagem, výslednou aplikaci zabalí do zip archivu s~příslušným názvem (FitGit-platforma-verze.zip, kde platforma je windows či linux a verze odpovídá názvu tagu) a nahrají na GitHub do již zmíněné sekce releases. Pokud během tohoto procesu jeden z~nástrojů selže, dá o~tom vědět vývojáři Aplikace například e-mailovou zprávou a archiv na GitHub nenahraje.

Pokud dojde k~nějakému selhání, oba nástroje poskytují detailní výpis, co se na nich spouštělo za příkazy a co vrátily za výsledek. Z~této zprávy je možné odvodit, co je se zdrojovými kódy špatně a opravit je.

Balíčky se vytváří pouze pro dvě platformy. V~průběhu vývoje Aplikace byly pokusy začlenit i 32bitové či 64bitové varianty a systém macOS, avšak větší počet různých balíčků s~různými specifiky vyžadoval vysoké nároky na údržbu skriptů starajících se o~jejich automatické vytváření. Z~časových důvodů od nich tedy bylo upuštěno.
