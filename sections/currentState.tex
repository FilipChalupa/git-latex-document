\chapter{Současný stav}
Jako ověření, že vytváření nové aplikace má význam, byly prozkoumány současné možnosti, které si kladou za cíl řešení stejného problému. K dispozici jsou například online nástroje ShareLaTeX a Overleaf. První zmíněný, „Online LaTeX editor snadný k použití s možností spolupráce více autorů“ {cit2}, se sice hned na úvodní stránce chlubí překladem do češtiny, který je ale nekompletní a práce s ním je tak často matoucí, z textů není jasné, co se zrovna děje, co se od uživatele očekává a jakým způsobem má pracovat. K tomu vyžaduje vysoký měsíční poplatek {cit3} za zpřístupnění většiny funkcí. Druhá služba, Overleaf, překlad do češtiny nenabízí vůbec, což není nutné považovat za nedostatek, protože požadavkem není řešení konkrétně pro české prostředí. Chybí však funkce pro procházení změn v historii projektu.

Obě služby trpí omezením, že se nelze dostat k úpravám projektu mimo ně. Uživatel nemůže tvořit ve svém oblíbeném editoru, musí si vystačit s online editorem, který navíc nenabízí žádnou offline variantu.

Alternativním přístupem je třeba verzovací systém Git, jež je zejména pro začátečníky s tímto systémem příliš komplikovaný.

těm psům
