\chapter{Úvod}
Častým scénářem při tvorbě různých skript, informačních podkladů, prezentací je spolupráce několika osob. V takových týmech je důležitá komunikace mezi jednotlivými členy a sdílení aktuálního rozpracovaného díla.

\section{Hypotéza}

K dispozici je spousta nástrojů, jichž smyslem je takovou tvorbu usnadnit; velké množství, různorodost, a v některých případech větší komplexita však značně komplikuje jejich nasazení. Důsledkem je uchýlení se k více známým a z podstaty primitivnějším řešením, například přeposílání rozdělané práce přes e-mail nebo předávání podkladů přes externí uložiště a nedělání záloh předešlých verzí. Tým tím ztrácí na efektivitě, což se může podepsat na kvalitě díla a zbytečném protahování tvorby.

\section{Terminologie}

\subsection{\LaTeX}

\uv{LaTeX je komplexní systém pro sazbu dokumentů. Stará se tedy o zalamování textu do odstavců, vkládání obrázků, seznamů, nadpisů, matematických výrazů, rozdělování dokumentu na jednotlivé stránky, ligatury, "vdovy" a "sirotky", odsazení\ldots} \cite{latex-def}

\subsection{Git}

Verzovací systém s podporou více uživatelů a ukládáním na vzdálený server například v internetu.

\subsection{Verzování}

Sledování projektu s průběžným ukládání změn pro zpětné dohledání přidaných či odebraných řádků v textovém souboru, případně přidání či odebrání binárního souboru.

\subsection{Kolaborace}

Spolupráce více osob na jednom projektu.

\subsection{Přístupnost}

Umožnění práce se složitějšími nástroji pro osoby s nedostatečnou znalostí ovládání.

\section{Struktura práce}

Následující kapitoly se budou zabývat ověřením hypotézy pomocí uživatelského průzkumu a analýzou existuících řešení, návrhem vlastního řešení s požadavky a volbou prostředků pro jeho implementaci, podrobnějším rozborem tvorby uživatelského rozhraní, distribucí aplikace, testováním, poznatky ohledně knihovny NodeGit, návrhy na další vývoj a závěrečným shrnutím.
